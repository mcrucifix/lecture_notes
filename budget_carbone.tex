\section{Budget carbone}

Les émissions de CO$_2$ peuvent s'exprimer en Tonnes de masse carbone, où en tonne de CO$_2$. 

La masse molaire du CO$_2$ est de 44, et la masse atomique du  carbone est de 12, donc,

\begin{equation}
  \text{1 Tonne de C}=\frac{44}{12} \text{Tonne de CO$_2$}
\end{equation}

A l'échelle de quelques décennies, les émissions de carbone sont cumulatives. La concentration en CO$_2$ dans l'atmosphère, mesurée en $ppmv$ (part par millions en volume) est proportionnelle aux émissions cumulées depuis le début de l'ère industrielle. 

On peut le voir de la façon suivante: 

La masse de l'atmosphère est de $5.3 \cdot 10^{18}$kg, cela représente environ $180 \cdot 10^{15}$ moles, en prenant une masse molaire moyenne de 28.8. 


Les émissions de CO$_2$ cumulées depuis l'ère industrielle sont de l'ordre de 700 GTC, ou encore $58 \cdot 10^{12}$ moles de CO$_2$. 

On considère qu'environ un tiers reste dans l'atmosphère, les deux autres tiers absorbés par la végétation et les océans (de moins en moins, cependant). 

\url{https://essd.copernicus.org/articles/15/5301/2023/}

et image sur:

\url{https://essd.copernicus.org/articles/15/5301/2023/essd-15-5301-2023-f03-web.png}

Ainsi, on attend un surplus d'environ $17.5 \cdot 10^{12}$ moles de CO$_2$ dans l'atmosphère.

Pour un gaz parfait, le rapport en volume correspond environ au rapport molaire. Il faut donc faire le rapport molaire pour estimer l'augmentation de la concentration en volume.  Ce rapport s'écrit: 

\begin{equation*}
  \frac{17.5\cdot 10^{12}  \ \text{moles de carbone}}{180 \cdot 10^{15} \  \text{moles d'atmoshpère}} = 107 \cdot 10^{-6} = 107 \ \text{ppmv}
\end{equation*}

Un peu plus de 100 part par million en volume, soit effectivement l'augmentation constatée depuis l'ère industrielle. 

Le \emph{forçage} radiatif est proportionnel au logarithme de la concentration en CO$_2$, et l'anomalie de température est proportionnelle au forçage radiatif. 

Différents effets non-linéaires se compensent, si bien qu'au final, à l'échelle de la centaine d'année, on trouve approximativement: 

\begin{equation*}
  \text{Réchauffement global} \propto \text{Émissions cumulées}
\end{equation*}

(le symbole $\propto$ signifie, en langage mathématique "proportionnel à") 

Selon cette relation de proportionalité, un réchauffement global de 2$^\circ$ correspond à des émissions cumulées de l'ordre de 1000 GTC. Soit, 300 GTC au delà des émissions déjà produites. C'est ce qu'on appelle le \emph{budget carbone} rapporté à une cible déterminée (ici, 2$^\circ$). 


Le budget carbone pour une limite de 2$^\circ$ réprésente 25 ans au rythme des émissions actuelles. 

Pour 1.5 degrés, on arriverait à 750 GTC, ce qui suggère que le quota est déjà quais atteint. Une littérature abondante tente de préciser ces chiffres, sans en changer fondamentalement les implications. 


\url{https://www.ipcc.ch/report/ar6/wg1/figures/chapter-5/figure-5-31}
