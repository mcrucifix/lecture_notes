% vim: set filetype=latex
% vim: set filetype=latex
\section{Sensibilit\'e climatique}

\subsection{Equilibre énergétique}

Dan ce court module nous établissons les bases de la théorie relative à la sensibilité climatique. 

La théorie part du principe d'\emph{équilibre radiatif}:

à l'équilibre: la Terre \emph{absorbe} un flux d'énergie solaire: environ 340 \wmm. L'équilibre est atteint lorsque, au sommet de l'atmosphère, la Terre \emph{émet} le même flux sous forme infrarouge. 

\url{https://www.ipcc.ch/report/ar6/wg1/figures/chapter-7/figure-7-2/}

L'équilibre est associé à une certaine température de surface, tel que les différents échanges radiatifs et convectifs au sein de l'atmosphère sont compatibles avec cet équilibre. 


\subsection{Forçage radiatif}

On appelle \emph{forçage radiatif} toute perturbation la composition chimique de l'atmosphère et de l'état de la surface qui modifie un de ces deux flux. 

Notamment, un \emph{doublement} de la concentration en dioxyde de carbone \emph{réduit} le flux infrarouge net au sommet emis au sommet de l'atmoshpère, d'environ 3.7\wmm. 


Si on considère comme réference l'an 1750, pour décrire l'an 2019, on trouve différentes sources de forçage radiatif associés aux activité humaines: 

\begin{itemize}
\item L'augmentation de la concentration en différents gaz à effet de serre, donc le CO$_2$, le CH$_4$, l'hemioxyde d'azote et les halogènes;
\item l'ozone troposphérique et stratosphérique;
\item les changements d'affectatien du sol;
\item les concentrations en aérosolse, notamment liés aux émissions de dioxye de souffre. 
\end{itemize}


Ces deux dernièrs forçages sont des forçages négatifs: ils augmentent le flux radiatif vers l'espace car ils contribuent à réfléchir davantage de lumière solaire vers l'espace. 

\url{https://www.ipcc.ch/report/ar6/wg1/figures/chapter-7/figure-7-6}

Une des difficultés de la gouvernance climatique est que ces différents forçages sont associés à des perturbation physiques qui n'évolueront pas selon les mêmes échelles de temps. 

Par exemple, une diminution des \emph{émissions} d'aérosols se traduit très rapidement par une diminution du forçage radiatif associé car les aérosols retombent rapidement sur la surface. 


En revanche, une diminution des émissions de \emph{CO$_2$} produiront une stagnation du forçage radiatif associé au CO$_2$, avec une diminution sur des échelles de plusieurs dizaines d'années. On dit que les émissions de CO$_2$ sont \emph{cumulatives}. 


\subsection{Réponse radiative}

A la suite du forçage, le système Terre réagit (répond)  en modifiant à son tour les flux radiatifs entrant et sortant. 

On a donc: 
\begin{equation}
  R_p = \lambda_p \Delta T
\end{equation}

\medskip

\begin{tabular}{ll}
  $F$: & le forçage \\
  $R$: & la réponse \\
  $N$: & le flux radiatif net 
\end{tabular}

\begin{equation}
  F + R = N
\end{equation}

La réponse se fait selon différentes modalités. Par exemple, au fur et à mesure que la température augmente, le flux radiatif émis augmente également du fait qu'une surface plus chaude émet davantage d'infrarouge. C'est la réponse de Planck. Si on définit $\Delta T$ la variation de température, on a

\begin{equation}
  R_p = \lambda_p \Delta T
\end{equation}

où on a utilisé l'indice $p$ pour désigner le phénomène de Planck. L'indice de proportionalité est \emph{négatif} dans la mesure où la réponse s'oppose au forçage de façon à le neutraliser : $\lambda_p < 0$. 

D'autres réponses ont lieu. 

Par exemple, avec le réchauffement, une partie de la glace fond, si bien que le flux solaire absorbé augmente. Cette fois l'indice de proportionalité est \emph{positif} dans la mesure où la fonte de la glace génère une perturbation visant à renforcer le forçage. 

\begin{equation}
  R_g = \lambda_g \Delta T \quad R_g>0
\end{equation}

On nomme \emph{feedbacks} les réponses radiatives associées avec les différentes composantes du système climatique. On dira qu'ils sont négatifs ou positifs selon le signe du coefficient $\lambda$. Ainsi, le feedback de Planck est un feedback \emph{negatif}, et celui associé à la glace est \emph{positif}. 

Certains feedbacks ont un signe incertain du fait de la complexité des mécanismes en jeu. Par exemple, les nuages peuvent contribuer à une réponse positive ou négative selon les nuages affectés et la façon dont ils sont affectés. 

Si on note, de façon générique $\lambda_i$ les différents coefficients, où $i\in\{p, g, \ldots\}$ pour $p$, Planck, $g$, glace, et tous les autres, alors on peut écrire 

\begin{equation}\label{eq:f}
  F + \sum \lambda_i \Delta T = N
\end{equation}

Lorsque l'équilibre est restauré (ce qui peu prendre plusieurs dizaines d'années, voir plus si on tient compte des composantes les plus lentes du système climatique), $N=0$. En résolvant l'équation \eqref{eq:f}, on trouve 

\begin{equation}
  \Delta T = - \frac{R}{\sum \lambda_i}
\end{equation}



Par convention, on appelle \emph{sensibilité climatique} le changement de température à l'équilibre lié à un doublement de la concentration en CO$_2$:

\def\cc{\ensuremath{2\times\text{CO}_2}}

\begin{equation}
  \Delta T_{\cc} = - \frac{R_{\cc}}{\sum \lambda_i}
\end{equation}

La valeur de $T_{\cc}$ est généralement estimée entre 2.5 et 4$^\circ$C. 

\url{https://www.ipcc.ch/report/ar6/wg1/figures/chapter-1/figure-1-16}


