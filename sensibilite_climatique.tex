% vim: set filetype=latex
% vim: set filetype=latex
\section{Sensibilit\'e climatique}

\subsection{Equilibre énergétique}

Dans ce court module nous établissons les bases de la théorie relative à la sensibilité climatique. 

La théorie part du principe d'\emph{équilibre radiatif}:
à l'équilibre:

\begin{itemize}
  \item la Terre \emph{absorbe} un flux d'énergie solaire: environ 340 \wmm. 
  \item L'équilibre est atteint lorsque, au sommet de l'atmosphère, la Terre \emph{émet} un flux de même intensité sous forme infrarouge. 
\end{itemize}

\url{https://www.ipcc.ch/report/ar6/wg1/figures/chapter-7/figure-7-2/}

L'équilibre est associé à une certaine température de surface, telle que les différents échanges radiatifs et convectifs au sein de l'atmosphère sont compatibles avec cet équilibre. 


\subsection{Forçage radiatif}

On appelle \emph{forçage radiatif} toute perturbation de la composition chimique de l'atmosphère ou de l'état de la surface qui modifie un de ces deux flux. 

Notamment, un \emph{doublement} de la concentration en dioxyde de carbone \emph{réduit} le flux infrarouge net au sommet emis au sommet de l'atmoshpère, d'environ 3.7\wmm. Ce forçage est, par convention, \emph{positif}. 


Si on considère comme référence l'an 1750, pour décrire l'an 2019, on trouve différentes sources de forçage radiatif associés aux activité humaines:  

\begin{itemize}
\item L'augmentation de la concentration en différents gaz à effet de serre, donc le CO$_2$, le CH$_4$, l'hemioxyde d'azote et les halogènes;
\item l'ozone troposphérique et stratosphérique;
\item les changements d'affectatien du sol;
\item les concentrations en aérosols, notamment  ceux liés aux émissions de dioxyde de soufre (combustion du charbon et du fuel lourd). Le dioxyde de soufre se transforme en acide sulfurique dans l'atmosphère où il se concentre sous la forme de goutelettes. 
\end{itemize}


Ces deux derniers forçages sont des forçages négatifs: ils augmentent le flux radiatif vers l'espace car ils contribuent à réfléchir davantage de lumière solaire vers l'espace. 

\url{https://www.ipcc.ch/report/ar6/wg1/figures/chapter-7/figure-7-6}

Le forçage lié aux aérosols est incertain car outre son action directe, assez facile à quantifier (réflexion de la lumière), il affecte le processus de formation et les propriétés optiques des nuages. Ce second processus, sans doute dominant, est beaucoup plus difficile à quantifier. 

Une des difficultés de la gouvernance climatique est que ces différents forçages sont associés à des perturbation physiques qui n'évolueront pas selon les mêmes échelles de temps. 

Par exemple, une diminution des \emph{émissions} d'aérosols se traduit très rapidement par une diminution du forçage radiatif associé car les aérosols retombent rapidement sur la surface. En améliorant la qualité de l'air rejeté par les centrales à charbon et les bateaux, on supprime un forçage négatif. 


En revanche, une diminution des émissions de CO$_2$ produiront une stagnation du forçage radiatif associé au CO$_2$ car celui-ci reste longtemps dans le système océan-atmosphère avant d'être digéré. On dit que les émissions de CO$_2$ sont \emph{cumulatives}. 

Le concentration de méthane (de l'ordre de 1900 part part milliard en volume, ppbv) est 210 fos plus petite que celle de CO$_2$ (415 ppm), mais le méthane est un puissant gaz a effet de serre si bien que le forçage radiatif associé aux anomalies de concentration de méthane depuis l'ère industrielle est seulement quatre fois plus faible (0.41 vs 2.16\ \wmm). 

Réduire les émissions de méthane conduit également rapidement à une diminution du forçage car le méthane est assez rapidement transformé en CO$_2$ par photolyse dans l'atmosphère (temps de demi-vie de l'ordre de 8 ans). Réduire les émissions de méthane peut donc être une stratégie à court terme pour soulager le réchauffement climatique. Si cependant cet action ne s'accompagne pas d'une réduction des émissions de CO$_2$, cette stratégie ne fait que postposer le caractère grave et irréversible du changement climatique. 

\subsection{Réponse radiative}

A la suite du forçage, le système Terre réagit (répond)  en modifiant à son tour les flux radiatifs entrant et sortant. 

Si on pose: 

\begin{tabular}{ll}
  $F$: & le forçage \\
  $R$: & la réponse \\
  $N$: & le flux radiatif net 
\end{tabular}

Le flux radiatif net est la somme du forçage et de la réponse. 

\begin{equation}
  F + R = N
\end{equation}

La réponse se fait selon différentes modalités. Par exemple, au fur et à mesure que la température augmente, le flux radiatif émis augmente également du fait qu'une surface plus chaude émet davantage d'infrarouge, selon le prinicipe du corps noir. C'est la réponse de Planck. Si on définit $\Delta T$ la variation de température moyenne globale, on a:

\begin{equation}
  R_p = \lambda_p \Delta T
\end{equation}

où on a utilisé l'indice $p$ pour désigner le phénomène de Planck. L'indice de proportionnalité est \emph{négatif} dans la mesure où la réponse s'oppose au forçage de façon à le neutraliser : $\lambda_p < 0$. 

D'autres réponses ont lieu. 

Par exemple, avec le réchauffement, une partie de la glace fond, si bien que le flux solaire absorbé augmente. Cette fois l'indice de proportionnalité est \emph{positif} dans la mesure où la fonte de la glace génère une perturbation visant à renforcer le forçage. 

\begin{equation}
  R_g = \lambda_g \Delta T \quad R_g>0
\end{equation}

On nomme \emph{feedbacks} les réponses radiatives associées aux différentes composantes du système climatique. On dira qu'ils sont négatifs ou positifs selon le signe du coefficient $\lambda$. Ainsi, le feedback de Planck est un feedback \emph{negatif}, et celui associé à la glace est \emph{positif}. 

La concentration en vapeur d'eau augmente dans l'atmosphère augmente également en fonction du réchauffement climatique et renforce le déficit radiatif. Il s'agit bien d'un \emph{feedback positif} qui est aujourd'hui bien quantifié et bien connu. 

Certains feedbacks ont un signe incertain du fait de la complexité des mécanismes en jeu. Par exemple, les nuages peuvent contribuer à une réponse positive ou négative selon les nuages affectés et la façon dont ils sont affectés. 

Si on note, de façon générique $\lambda_i$ les différents coefficients, où $i\in\{p, g, v, n,  \ldots\}$ pour $p$, Planck, $g$, glace, $v$ vapeur d'eau, $n$, nuages,  et tous les autres, alors on peut écrire 

\begin{equation}\label{eq:f}
  F + \sum \lambda_i \Delta T = N
\end{equation}

Lorsque l'équilibre est restauré (ce qui peu prendre plusieurs dizaines d'années, voir plus si on tient compte des composantes les plus lentes du système climatique), $N=0$. En résolvant l'équation \eqref{eq:f}, on trouve 

\begin{equation}
  \Delta T = - \frac{F}{\sum \lambda_i}
\end{equation}



Par convention, on appelle \emph{sensibilité climatique} le changement de température à l'équilibre lié à un doublement de la concentration en CO$_2$:

\def\cc{\ensuremath{2\times\text{CO}_2}}

\begin{equation}
  \Delta T_{\cc} = - \frac{F_{\cc}}{\sum \lambda_i}
\end{equation}

La valeur de $T_{\cc}$ est généralement estimée entre 2.5 et 4$^\circ$C. Cette estimation repose différentes stratégies: 

\begin{itemize}
  \item Modélisation au moyen de modèles de circulation générale;
  \item Détection de relations entre sensibilité climatique et caractéristiques associées à la variabilité climatique (stratégie des contraintes émergentes);
  \item Estimation du forçage radiatif et de la réponse associé à des changements climatiques passés, par exemple le dernier maximum glaciaire, le Pliocène ou l'Éocène.
\end{itemize}

\url{https://www.ipcc.ch/report/ar6/wg1/figures/chapter-1/figure-1-16}

Cela signifie donc que, sans même tenir compte des autres gaz à effet de serre, on attend un réchauffement climatique de l'ordre de 3 degrés pour une concentration en CO$_2$ de $280\text{ppm} \times 2=560$ppm (280\ ppm est la référence pré-industrielle). 

Nous sommes plus qu'à  mi-chemin ($280 \times \sqrt 2 = 395\ $ppm) qui correspond à 1.5$^\circ$. Les forçage positifs (méthane), et négatifs (aérosols) additionnels tendent aujourd'hui à se compenser. 


